\documentclass[a4paper, 12pt]{article}
% Title information
\title{Journal de laboratoire}
\author{Arian Dervishaj}
\date{\today}

\begin{document}
\maketitle
\pagebreak

\section*{3. Exercices}
\subsection*{3.1 Mesure avec resistance}
\begin{enumerate}
    \item 97.0 $\Omega$
    \item \begin{itemize}
        \item $U = R*I \Longleftrightarrow I = U/R \Longleftrightarrow I = 5.00 / 97.0 = 0.052 A = 52 mA $
        \item $P = U * I = 5 * 0.052 = 0.26W$
        \end{itemize} 
    \item[4] La tension mesurée aux bornes de la resistance est de $4.88V$
    \item[5] La mesure du courant est de $48 mA$
\end{enumerate}

\subsection*{3.2 Résistance en série}
\begin{enumerate}
    \item Les deux sont de $97 \Omega$
    \item Oui
    \item 1ere : 23mA, 2ème : 23 mA
    \item \begin{itemize}
        \item $I = U/R \Longleftrightarrow I = 5.00 / 97.0 = 0.052 A = 52 mA $
        \item $P = U * I = 5 * 0.052 = 0.26W.$
        \end{itemize} 
\end{enumerate}

\subsection*{3.3 Résistance en parallèle}
\begin{enumerate}
    \item $97 \Omega$
    \item Le tension aux bornes de la première resistance est de 4.90V et la deuxième est de 4.93V
    \item $R_1 = 47.8mA$, $R_2 = 47.2mA$
    \item 
        \begin{itemize}
            \item $I = U/R \Longleftrightarrow I = 5.00 / 97.0 = 0.052 A = 52 mA$
            \item $P = U * I = 5 * 0.052 = 0.26W.$
        \end{itemize}
    \item Le courant augmente en ajoutant des resistances.
\end{enumerate}

\subsection*{3.4 Diode électroluminescente}
\begin{enumerate}
    \item 
\end{enumerate}
\end{document}
