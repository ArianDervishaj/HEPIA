\documentclass[a4paper, 12pt]{article}
\usepackage{amsmath}
\usepackage{array}
% Title information
\title{Labo 1 : Représentation de l'information}
\author{Arian Dervishaj}
\date{\today}

\begin{document}

\maketitle
\section*{1. Réponse}
\subsection*{1.2 Analyse}
\begin{enumerate}
    \item 128 en little endian\\ 
          2'147'483'648 en big endian
    \item Little endian
    \item 129 bytes
    \item 16x32 $\rightarrow$ 512
    \item 4 bits
    \item 4 * 512 = 2048

\end{enumerate}

\subsection*{1.3 Réparation d'un fichier BMP}
\begin{enumerate}
  \item[] 3. 01 $\rightarrow$ noir, 23 $\rightarrow$ marron etc etc

\end{enumerate}

\section*{2. Format d'un fichier audio WAV}
\subsection*{2.2 Analyse du contenu d'un fichier WAV}

  \begin{tabular}{|c|c|c|}
    \hline
    Bytes & Description & Valeur (ASCII ou décimal) \\
    \hline
    52 49 46 46 & Chunk ID & RIFF \\
    \hline
    F5 56 00 00 & Taille restante du fichier & $\begin{aligned} 
    & 15 \times 16 + 5 \\
    & + 5 \times 16^{3} + 6 \times 16^{2} \\
    &= 22'261 
    \end{aligned}$ \\
    \hline
    57 41 56 45 & Format (RIFF Type) & WAVE \\
    \hline
    66 6D 74 20 & Sub-Chunk format ID & fmt \\
    \hline
    10 00 00 00 & Taille restante du sub-chunk fmt & $1 \times 16 = 16$ \\
    \hline
    01 00 & Format audio & 1 = PCM \\
    \hline
    01 00 & NumChannels & Mono \\
    \hline
    11 2B 00 00 & SampleRate & $\begin{aligned} 
    & 1 \times 16 + 1 \\
    & + 2 \times 16^{3} + 11 \times 16^{2} \\
    &= 1'611'025 
    \end{aligned}$ \\
    \hline
    11 2B 00 00 & ByteRate & $\dfrac{1'611'025 \times 1 \times 8}{8} = 1'611'025$ \\
    \hline
    01 00 & ByteBloc & $\dfrac{1 \times 8}{8} = 1$ \\
    \hline
    08 00 & BitsPerSample & 8 \\
    \hline
    64 61 74 61 & Sub-chunk data ID & data \\
    \hline
    D1 56 00 00 & Nombre d'octets de données & 22'225 \\
    \hline
    -- & Premier échantillon & 80 9F BD D6 EB F8 FE FC \\
    \hline
    -- & Deuxième échantillon & F2 E1 CA AE 8F 70 51 35 \\
    \hline
    -- & Troisième échantillon & 1E 0D 03 01 07 14 29 42 \\
    \hline
  \end{tabular}

\end{document}
\begin{enumerate}
  \item[] 8. sinusoidale
\end{enumerate}

\subsection*{2.3 Répparation d'un fichier WAV}
\begin{enumerate}
  \item[] 4. Oui

\end{enumerate}

\end{document}
