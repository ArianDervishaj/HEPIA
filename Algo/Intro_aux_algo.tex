\documentclass[12pt]{article}
\usepackage[a4paper, margin=2cm]{geometry}
\setlength{\parindent}{0pt}
\usepackage[linesnumbered,ruled,vlined]{algorithm2e}

% Title information
\title{%
  Introductions aux algorithmes \\
  \large Notes de cours}
\author{Arian Dervishaj}
\date{\today}

\begin{document}
\maketitle
\pagebreak

\section{Notions de base d'algorithmique}
\begin{algorithm}
    \caption{Nombre premier}
    \KwData{Deux nombres entiers positifs, $div$ et $nb$.\\ Un boolean, $estPremier$}
    \KwResult{$nb$ est premier : True ou False}
    
    $nb \leftarrow Input$ \\
    $estPremier \leftarrow True$ \\
    $div \leftarrow 2$ \\
    \While{$div < \sqrt[2]{nb}$}{
        \If{$nb \bmod div = 0$}{
            $estPremier \leftarrow False$
        }
        $div \leftarrow div + 1$ \\
    } 
    \Return $estPremier$\;
\end{algorithm}

\begin{algorithm}
    \caption{Calcul de factorielle}
    \KwData{Un nombre entier positif, $nb$}
    \KwResult{$res$ : résultat de la factorielle}

    $nb \leftarrow Input$ \\
    $res = 1;$ \\
    \For{$int\ i = 1, i <= nb, i++$}{
     $res = res * i;$
    }
    \Return $res$\;
\end{algorithm}

\begin{algorithm}
    \caption{Trouver le plus petit entier d'une liste}
    \KwData{$nb[\ ]$, liste d'entier}
    \KwResult{$minNb$ : plus petit entier de la liste}

    $nb[\ ] \leftarrow Input$ \\
    $size \leftarrow$ longeur de nb[\ ]\\
    $minNb \leftarrow nb[0]$ \\
    $index \leftarrow 0$ \\

    \For{$int\ i = 1, i <= nb, i++$}{
     \If{$nb[i] < nb[index]$}{
        $minNb \leftarrow nb[i]$ \\
        $index \leftarrow i$ \\
     }
     }
    \Return $minNb$\;

\end{algorithm}

\end{document}
