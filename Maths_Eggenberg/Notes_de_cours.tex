\documentclass[12pt]{article}
\usepackage[a4paper, margin=2cm]{geometry}
\usepackage{xcolor}
\usepackage{multicol}
\setlength{\parindent}{0pt}
% Title information
\title{Maths Eggenberg}
\author{Arian Dervishaj}
\date{\today}

\begin{document}
\maketitle
\pagebreak

\section*{Conversion}
\subsection*{Méthode de la soustraction}

\begin{enumerate}
    \item   $(78)_{10}= 64 + (78-64) = 64 + 8 + 6 = 64 + 8 + 4 + 2 = \\
            2^{6} + 2^{3} + 2^{2} + 2^{1} = (1001110)_2$
    \item   $(7904)_{10} = 2 * 60^{2} + 11 * 60^{1} + 44 * 60^{0} = (021144)_{60}$
\end{enumerate}

\subsection*{Méthode de la division}

\begin{enumerate}
    \item   $7904 / 60 = 131\  r\ 44 \\
            131 / 60 = 2\ r\ 11 \\ 
            2 / 60 = 0\ r\ 2 \\ 
            (7904)_{10} = (021144)_{60}$
\end{enumerate}

\subsection*{Exercice}
    $(07211403)_{23} = 7 * 23^{3} + 21 * 23^{2} + 13*23 + 3 = (96'603)_{10} = (03071607)_{31}$


\section*{Représentation des entiers signées}
\subsection*{Complément à base deux}
\begin{itemize}
    \item Si $X \ge 0 \rightarrow$ 1er bit est 0, X s'écrit sur les N-1 bits restants
    \item Si $X \le 0 \rightarrow$ 1er bit est 1, X s'écrit sur les N-1 bits restants
\end{itemize}

\textbf{Notation : } Si X est exprimé en compéément à deux sur N bits, on notera $(X)_{\overline{2}^{N}}$

\subsection*{Conversion simplifiée}
Exemple : $-72 = (?)_{\overline{2}^{8}}$
\begin{enumerate}
    \item Convertir 72 en base 2 : $(01001000)_{2}$
    \item Inverser les bits : $(10110111)_{2}$
    \item Ajouter 1 en binaire : $(10111000)_{2}$
    \item Et donc : $-72 = (10111000)_{\overline{2}^{8}}$
\end{enumerate}

\subsection*{Reconversion en base de 10}
\begin{itemize}
    \item Si 1er bit = 0 $\rightarrow X = (...)_2$
    \item Si 1er bit = 1 $\rightarrow X = -2^{N-1} + (...)_2$
\end{itemize}
$(...)_2 = N-1$ bits restants en base 2

\subsection*{Attention aux overflow}
\begin{itemize}
    \item \qquad $(5)_{10}$ \qquad $(0101)_{\overline{2}^8}$
    \item + \quad $(5)_{10}$ \qquad $(0101)_{\overline{2}^8}$
    \item = \quad $(10)_{10}$ \quad \  $(1010)_{\overline{2}^8}$
    \item Alors que : $(1010)_{\overline{2}^8} = -2^{3} + (010)_{2} = -8 + 2 = -6$
\end{itemize}

\section*{Binaire à virgule}
On ne peut que travailler avec des nombres avec des décimales \textbf{FINIES}.

Que vaut $(13.625)_{10} = (?)_{2}$
\subsubsection*{Methode par tatonnement}
$ \rightarrow 13.625 = 8 + 4 + 1 + \frac{1}{2} + \frac{0}{4} + \frac{1}{8} = (1101,101)_2$

\subsubsection*{Methode par multiplication}

$ 13 = 2^{3} + 2^{2} + 2^{0} = 1101 \\
(0.625)_{10} = (?)_2 \rightarrow \frac{2*0.625}{2} = \frac{1.25}{2} = \frac{1}{2} + \frac{0.25}{2} \\
\frac{0.25}{2} = \frac{0.5}{4} = \frac{0}{4} + \frac{0.5}{4} \\
\frac{0.5}{4} = \frac{1}{8} + 0$
\vspace{10pt}

Partie entière : bit à garder. \\
Recommencer avec la partie décimal. \\
(Garder les bit dans l'ordre d'apparition.)

\section*{Nombres à virgule flottante}

\subsection*{Ecriture scientifique : en base 10}

$\pm x,y * 10^{Z}$ \quad $ z \in Z, y \in N, x \in \{1,\dots,9\}$

\subsection*{Encodage : En binaire}


\colorbox{yellow!20}{$x_0$} \colorbox{red!20}{$x_1 \dots x_8$} \colorbox{green!20}{$x_9 \dots x_{31}$}
\begin{itemize}
    \item \colorbox{red!20}{Exposant}
    \item \colorbox{green!20}{Mantisse (23bits)}
\end{itemize}
$(-1)^{x_0}(2)^{(x_1 \dots x_8)_2 - 127}(1,x_9 \dots x_{31})_2$

\subsubsection*{En 32 bits}
    \begin{itemize}
        \item 11 bits pour l'exposants
        \item 23 bits pour la Mantisse
        \item 127 de décalage
    \end{itemize}
\subsubsection*{En 64 bits}
    \begin{itemize}
        \item 11 bits pour l'exposants
        \item 52 bits pour la Mantisse
        \item 1023 de décalage
\end{itemize}

\subsection*{Exemple : }

\subsubsection*{1.} 
$(1 \quad 11001011 \quad 0010...0)_{float} = $
$(-1)*2^{203-127} * (1,\frac{1}{8}) = 1.125 * 2^{76}$

\subsubsection*{2.}
$(-784)_{10} = (?)_{float}$
$784 = 2^{9}+2^{8}+2^{4} = 1 \colorbox{blue!20}{10001 0000} $\\
$ e = 127 + 9 = 136 = 128 + 8 =  \colorbox{red!20}{1000 1000}$ 

$(1 | \colorbox{red!20}{1000 1000} | \colorbox{blue!20}{10001 0000} \overbrace{0 \dots 0}^{14})_{float} = (-784)_{10}$

\subsubsection*{3.}
$(0.07)_{10} = (0,00\colorbox{blue!20}{1000111010010100001})_2 = (1,0)*2^{-4}$ \\
$e = 127 -4 = 123 = 64 + 32 + 16 + 8 + 2 + 1 = \colorbox{red!20}{1111011}$

$(0 \colorbox{red!20}{1111011} \colorbox{blue!20}{1000111010010100001}00000)_{float}= (0.07)_{10}$ 

\end{document}
