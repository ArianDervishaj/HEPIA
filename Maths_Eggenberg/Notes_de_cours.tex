\documentclass[a4paper, 12pt]{article}
% Title information
\title{Maths Eggenberg}
\author{Arian Dervishaj}
\date{\today}

\begin{document}
\maketitle
\pagebreak

\subsection*{Méthode de la soustraction}

\begin{enumerate}
    \item   $(78)_{10}= 64 + (78-64) = 64 + 8 + 6 = 64 + 8 + 4 + 2 = \\
            2^{6} + 2^{3} + 2^{2} + 2^{1} = (1001110)_2$
    \item   $(7904)_{10} = 2 * 60^{2} + 11 * 60^{1} + 44 * 60^{0} = (021144)_{60}$
\end{enumerate}

\subsection*{Méthode de la division}

\begin{enumerate}
    \item   $7904 / 60 = 131\  r\ 44 \\
            131 / 60 = 2\ r\ 11 \\ 
            2 / 60 = 0\ r\ 2 \\ 
            (7904)_{10} = (021144)_{60}$
\end{enumerate}

\subsection*{Exercice}
    $(07211403)_{23} = 7 * 23^{3} + 21 * 23^{2} + 13*23 + 3 = (96'603)_{10} = (03071607)_{31}$



\pagebreak
\subsection*{Binaire à virgule}
Que vaut $(13.625)_{10} = (?)_{2}$
\subsubsection*{Methode 1}
$ \rightarrow 8 + 4 + 1 + \frac{1}{2} + \frac{0}{4} + \frac{1}{8} = (1101,101)_2$

\subsubsection*{Methode 2}

$\frac{2*0.625}{2} = \frac{1.25}{2} = \frac{1}{2} + \frac{0.25}{2} \\
\frac{0.25}{2} = \frac{0.5}{4} = \frac{0}{4} + \frac{0.5}{4} \\
\frac{0.5}{4} = \frac{1}{8} + 0$ \\

Quand on a 1 au dessus de la fraction $\rightarrow$ ajoute 1 dans le binaire



\end{document}
